\subsubsection{Strategy}
Strategy games creates a world that forces the players to think and achieve victory through planning. The main difference between puzzle games and strategy games is that in strategy games the player has opponents to battle against\cite{FundamentalsOfGameDesign}. "This definition distinguishes strategy games from puzzle games that call for planning in the absence of conflict, and from competitive construction and management
simulations that require planning but not direct action against an opponent."\cite{FundamentalsOfGameDesign}. Generally, strategy games use low-quality game elements to be able to have large quantities of elements in the scene and terrains usually use heightmap\footnote{"Height mapping (also known as parallax mapping) is a similar concept to normal mapping, however this technique is more complex - and therefore also more performance-expensive. Heightmaps are usually used in conjunction with normalmaps, and often they are used to give extra definition to surfaces where the texture maps are responsible for rendering large bumps and protrusions."\cite{UnityHeightMap}} based terrains\cite{GameEngineArchitecture}. Some famous strategy games are StarCraft, Red Alert, Stellaris, Civilization  and Total War\cite{StrategyGames}.

\begin{figure}[!h] \centering \includegraphics[width=2.5in]{heightmap_a} \caption{"This heightmap represents a 19.5km x 19.5km square."\cite{HeightmapRenderingUsingAFloorcastingAlgorithm}} \label{fig:heightmap-a} \end{figure}

\begin{figure}[!h] \centering \includegraphics[width=2.5in]{heightmap_b} \caption{A 3D representation of the heightmap presented in figure \ref{fig:heightmap-a} \cite{HeightmapRenderingUsingAFloorcastingAlgorithm}} \label{fig:heightmap-b} \end{figure}