\subsubsection{Racing}

\begin{figure}[!h] \centering \includegraphics[width=2.5in]{tree_a} \caption{"Opacity maps are popular for making layered billboard tree geometry. Use enough of
these and you’ll have a convincing forest if the camera is distant enough"\cite{AnIntroductionToComputerGraphicsForArtists}} \label{fig:racing-tree} \end{figure}

Racing games can be divided into two different sub-genres, the first one are the games that focus mainly on simulating a race and the other one are games that are not entirely focused on races but have vehicle mechanics as well. Apart from a different kind of gameplay, these games share a common thread, which is the physics-based simulation of racing\cite{AIGameEngineProgramming}. Some of the techniques used in racing game engines are rendering 2D elements for distant objects like tree as shown in figure \ref{fig:racing-tree}, hills and mountains, that why those games can improve their performance, one other technique divides the game world into 2D sectors, this can be used to improve performance when rendering elements and using artificial intelligence like pathfinders\cite{GameEngineArchitecture} as shown in figures \ref{fig:racing-path-a} and \ref{fig:racing-path-b}. Some famous racing game series are: Forza, Gran Turismo, Need for Speed, Mario Kart and Dirty\cite{RacingGames}.

\begin{figure}[!h] \centering \includegraphics[width=2.5in]{race_a} \caption{"Representation of a Racing Line"\cite{SearchingForTheOptimalRacingLineUsingGeneticAlgorithms}} \label{fig:racing-path-a} \end{figure}

\begin{figure}[!h] \centering \includegraphics[width=2.5in]{race_b} \caption{"An example of track decomposition process based on the intersection between the shortest path SP (reported as a blue line) and the minimum curvature path MCP (reported as a red line)."\cite{SearchingForTheOptimalRacingLineUsingGeneticAlgorithms}} \label{fig:racing-path-b} \end{figure}