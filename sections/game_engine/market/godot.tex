\subsubsection{Godot}
Godot is a free and open source\footnote{"The fundamental purpose of open source licensing is to deny anybody the right to exclusively exploit a work. Typically, in order to permit their works to reach a broad audience, and, incidentally, to make some sort of living from making works, creators are required to surrender all, or substantially all, of the rights granted by copyright to those entities that are capable of distributing and thereby exploiting that work."\cite{OpenSource}} game engine currently under the MIT licence\footnote{"The MIT License is short and to the point. It lets people do almost anything they want with your project, like making and distributing closed source versions."\cite{MIT}}. The engine focus both in 2D and 3D games and has support for several programming languages such as: C++, C\#, Rust, Nim, D and GDScript\footnote{"GDScript is a high level, dynamically typed programming language used to create content. It uses a syntax similar to Python (blocks are indent-based and many keywords are similar). Its goal is to be optimized for and tightly integrated with Godot Engine, allowing great flexibility for content creation and integration."\cite{GDScript}}. The engine can also compile for several platforms such: Microsoft Windows, Linux, Mac OS X, Android, iPhone/iPad and HTLM5\cite{GodotAbout} and some games developed using the engine were: Tanks of Freedom, Stereobreak, Steno Arcade, DeepSixed,Seek Etyliv, Grimante and Shipwreck\cite{GodotGames}.