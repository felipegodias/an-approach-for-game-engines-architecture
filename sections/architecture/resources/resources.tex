\subsection{Resources}
Games need a bunch of elements that are specific for each kind of game, these elements can be called resources or even assets. Some of theses assets are 3D models, material\footnote{"In the real world, each object reacts differently to light. Steel objects are often shinier than a clay vase for example and a wooden container does not react the same to light as a steel container. Each object also responds differently to specular highlights. Some objects reflect the light without too much scattering resulting in a small highlights and others scatter a lot giving the highlight a larger radius."\cite{LearnOpenGL}}, images, shaders\footnote{"The current state of the art in real-time computer graphics is programmable shading."\cite{OpenGLBible}}, animations and much more\cite{UnityResources}. It's a game engine responsibility to manage all those resources in run-time. Some game engines creates it's own asset file format, this way the engine can load the assets faster, since it's an already known format. In order to this work, the game engine needs to compute these assets from native formats (e.g. .PNG for images, .OBJ for 3D models) to a internal one during the game development. The resource manager must ensures that assets only have one unique instance in memory, must control resources lifetime and identify when each asset can be safely disposed. One big responsibility is to keep referential integrity, so if assets reference one another, they can be unloaded then reloaded and the game must be like the initial state\cite{GameEngineArchitecture}.