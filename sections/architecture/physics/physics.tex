\subsection{Physics}
Just like the Rendering system, the Physics simulation has a world for itself and most AAA\footnote{AAA is oftenly used in game industry to represent high budget games.} games have a physics simulation in it, so for achieving a good quality game, a physics simulation must be integrated within the engine\cite{GamePhysicsEngineDevelopment}. Some features that a game engine with physics simulation should provide are a collision detection between static and dynamic scene objects, simulate rigid bodies over the influence of external forces and gravity, string-mass simulation, ray and shape casts, volume triggers, vehicle simulation with realistic suspensions, rag dolls and much more\cite{GameEngineArchitecture}. Since a physics system could be considered a whole project for itself, when creating a game engine you should consider using already existing physics engines like Box2D, PhysX\footnote{"The NVIDIA PhysX SDK is a scalable multi-platform physics solution supporting a wide range of devices, from smartphones to high-end multicore CPUs and GPUs. PhysX is already integrated into some of the most popular game engines, including Unreal Engine, and Unity3D."\cite{PhysXAbout}} or Havok\footnote{"Havok Physics offers the fastest, most robust collision detection and physical simulation technology available, which is why it has become the gold standard within the games industry and has been used by more than half of the top selling titles this console generation."\cite{HavokAbout}}\cite{PhysicsForGameDevelopers}. So the game engine needs to integrate one or many of those physics engines with other services such as the scene manager to know where which object is and updates its position and the render manager to be able to debug physics data.

\begin{figure}[!h] \centering \includegraphics[width=2.5in]{physics-contact} \caption{2D Physics contact points and circle move direction\cite{GameEngineArchitecture}.} \label{fig:physics-contact} \end{figure}