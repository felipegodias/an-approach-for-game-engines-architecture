\subsection{Game Loop}
Just like real life, time has great importance for games as well. In a game there's too many things happening at once, so it's the game loop manager responsibility to keep all of those subsystems synchronized\cite{GameEngineArchitecture, GameProgrammingPatterns}. Scenes behaviours can be updated every game loop differently from rendering that must keep synchronized with the user's monitor (e.g. 30hz or 60hz). Physics simulation needs a more frequent update (e.g. 120hz) from the rendering to keep the simulation more fluid\cite{GameEngineArchitecture}. In a naive approach, game loop manager also known as simulation manager can be easily implemented using the programming while operation, while the game is running, all subsystems will be updated according to its priority, generally, the priority order is input manager, scene manager, physics manager and render manager\cite{GameProgrammingPatterns} an example can be seen in figure \ref{fig:game-loop}.

\begin{figure}[!h] \centering \includegraphics[width=3in]{game-loop} \caption{A game loop example diagram.\cite{GameProgrammingPatterns}.} \label{fig:game-loop} \end{figure}